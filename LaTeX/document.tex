\documentclass{perassignments}

\usepackage{amsmath}
\usepackage{amssymb}
\usepackage{mathtools}
\usepackage{xcolor}
\usepackage{enumitem}
\usepackage{intcalc}
\usepackage{perheader}
\usepackage[abjad]{pertheorems}
\usepackage{common}
\usepackage{listings}
\usepackage{graphicx}
\usepackage{float}
\usepackage{multirow}
\usepackage{hyperref}
\usepackage{polyglossia}

\setdefaultlanguage{persian}


\usepackage{xepersian}
\settextfont{XBNiloofar}
\setmathdigitfont{XBTabriz}
\allowdisplaybreaks
\begin{document}
در مدل طلایی برای ضرب و جمع اعشاری IEEE754 از سایت
\lr{http://weitz.de/ieee}
استفاده کردیم. برای ضرب ماترسی ابتدا دو ماتریس که درایه‌هایشان به طور تصادفی از اعداد بین دو عدد
   \(\text{\lr{0x3A83126F}} = 0.001\)
   و
 \({\text{\lr{0x42C80000}} = 100.0}\)
 انتخاب می‌شوند درست کردیم و سپس با الگوریتم معمولی ضرب ماتریسی و ماژول‌های ضرب و جمع اعشاری  این دو ماتریس را در هم ضرب کردیم. در نهایت مقادیر دو ماتریس ورودی در فایل
\lr{ memory\_tb\_init.txt} 
قرار می‌دهیم و ماتریس جواب را در فایل
\lr{processor\_tb\_check.txt}
 را بارگذاری می‌کنیم تا با مقادیر تولید شده توسط سخت افزار مقایسه کنیم. 
 
\begin{center}
	\begin{tabular}{|p{3cm}|p{9cm}|p{2cm}|}
		\hline
		اعضای گروه & وظایف & درصد مشارکت 
		\\ \hline
		محمد جواد هزاره & طراحی
		\lr{Main Control Unit}
		 و تست‌بنچ، تکمیل طراحی
		\lr{Top module}
		 و تست‌بنچ. & 
		 \lr{20\%} 

		\\ \hline
		مازیار شمسی‌پور & 
		طراحی
		\lr{Control Unit} 
		و تست‌بنچ، طراحی 
		\lr{Top Module}
		و تست‌بنچ. تکمیل گزارش &
	  \lr{20\%} 
		\\ \hline
		عماد زین‌اوقلی & طراحی 
		\lr{Memory} 
		و 
		\lr{Arbiter}
		و تست‌بنچ، تکمیل طراحی 
		\lr{Matrix Multiplier}
		و تست‌بنچ. &
		 \lr{20\%} 
		\\ \hline 
		پویا یوسفی &  طراحی 
		\lr{Golden Model}
		و انجام عملیات سنتز. تکمیل گزارش &
		 \lr{20\%} 
		\\ \hline 
		بردیا محمدی & طراحی 
		\lr{Matrix Multiplier}
		و تست‌بنچ. تکمیل گزارش&
		 \lr{20\%} 
		\\ \hline
		
	\end{tabular}
\end{center}
\end{document}

